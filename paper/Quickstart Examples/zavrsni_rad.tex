\documentclass[zavrsnirad]{../fer}
% Dodaj opciju upload za generiranje konačne verzije koja se učitava na FERWeb
% Add the option upload to generate the final version which is uploaded to FERWeb


\usepackage{blindtext}
\usepackage{subcaption}
\newtheorem{teorem}{Teorem}

%--- PODACI O RADU / THESIS INFORMATION ----------------------------------------

% Naslov na engleskom jeziku / Title in English
\title{Galjerkin finite element method}

% Naslov na hrvatskom jeziku / Title in Croatian
\naslov{Galjorkinova metoda konačnih elemenata}

% Broj rada / Thesis number
\brojrada{1884}

% Autor / Author
\author{Hrvoje Radoš}

% Mentor 
\mentor{prof. dr. sc. Tomislav Burić}

% Datum rada na engleskom jeziku / Date in English
\date{June, 2025}

% Datum rada na hrvatskom jeziku / Date in Croatian
\datum{lipanj, 2025.}

%-------------------------------------------------------------------------------


\begin{document}


% Naslovnica se automatski generira / Titlepage is automatically generated
\maketitle


%--- ZADATAK / THESIS ASSIGNMENT -----------------------------------------------

% Zadatak se ubacuje iz vanjske datoteke / Thesis assignment is included from external file
% Upiši ime PDF datoteke preuzete s FERWeb-a / Enter the filename of the PDF downloaded from FERWeb
\zadatak{Files/zadatak.pdf}


%--- ZAHVALE / ACKNOWLEDGMENT --------------------------------------------------

\begin{zahvale}
	% Ovdje upišite zahvale / Write in the acknowledgment
	Hvala na kavi...
\end{zahvale}


% Odovud započinje numeriranje stranica / Page numbering starts from here
\mainmatter


% Sadržaj se automatski generira / Table of contents is automatically generated
\tableofcontents


%--- UVOD / INTRODUCTION -------------------------------------------------------
\chapter{Uvod}
\label{pog:uvod}
Diferencijalne jednadžbe su jedan od temeljnih matematičkih
alata koji se koriste za matematičko modeliranje. Njihova
važnost se obično prvo primijeti u fizici gdje se one vrlo
intuitivno pojavljuju u područjima kao što su klasična mehanika,
elektromagnetizam, termodinamika pa i malo manje intuitivno u
područjima kao što su kvantna mehanika. Diferencijalne jednadžbe
imaju velik utjecaj i na područja izvan fizike.
S njima se mogu modelirati izmjene populacija, širenje zaraza,
oblikovanje raznih genetskih obilježja (uzoraka),
širenje signala u neuronima, procesiranju slika, raznim
modeliranjima tržišta itd.
\bigskip
\\
Diferencijalne jednadžbe uglavnom nije teško tumačiti. Najveći
izazov kod rada s njima je svakako traženje rješenja.
Rješenje ne mora uvijek postojati, ne mora biti jedinstveno,
a ako postoji i jedinstveno je ne mora biti moguće zapisati
ga analitički. Tim problemima se između ostaloga bavi numerička
matematika. Ovaj rad se bavi linearnim parcijalnim diferencijalnim
jednadžbama s konstantnim koeficijentima i rubnim uvjetom odnosno
traženjem njihova rješenja koristeći Galjorkinovu metodu
konačnih elemenata. Cilj rada je napraviti
C++ biblioteku koja služi kao efikasan, ali iznad svega
jednostavan alat za rješavanje PDJ-ova.
% Neki od radova koje ćemo citirati su \cite{6248073,6247753,ghiglia_pritt_phase_unwrapping,hartley2003multiple,4250461,123DCatch}.


%-------------------------------------------------------------------------------
\chapter{Glavni dio}
\label{pog:glavni_dio}
Galjorkinova metoda konačnih elemenata je složen proces
koji se sastoji od više dijelova. Na početku ovog 
poglavlja nudim kratki opis kako ovaj proces izgleda, a
u narednim poglavljima ulazim u detalje.
\bigskip
\\ 
Jednadžba koju rješavam na domeni $\Omega \subset \mathbb{R}^2$ je:

\begin{equation}
	\label{PDJ}
  C_0 \frac{\partial^2 u}{\partial x^2}
	+ C_1 \frac{\partial^2 u}{\partial x \partial y}
	+ C_2 \frac{\partial^2 u}{\partial y^2}
	+ C_3 \frac{\partial u}{\partial x}
	+ C_4 \frac{\partial u}{\partial y}
	+ C_5 u = f(x,y)
  \quad u \in C^2(\Omega)
\end{equation}

Ključna ideja ove metode je particioniranje domene na
što sitnije odnosno finije dijelove koje nazivamo elementima.
Takvu particiju nazivamo mesh. S meshom potom definiramo
bazne funkcije koje čine bazu s kojom ćemo interpolirati
rješenje koje tražimo. Dodatnim raspisivanjem ćemo dobiti
linearni sustav čijim rješavanjem dobivamo naše traženo
rješenje.
\bigskip
\\ 
\label{uvjetNaf}
Postoji par napomena koje valja primijetiti. Ako obratimo
pažnju na jednadžbu \eqref{PDJ}, možemo primijetiti da smo
se ograničili na rješenja iz prostora $C^2(D)$ dodatnom analizom
možemo zaključiti i da funkcija $f$ mora biti neprekinuta.
Ovo se na prvu iz perspektive primjene ne čini kao velika 
restrikcija jer nekako očekujemo da će fizikalne veličine u 
stvarnom životu uvijek biti neprekinute, ali ne samo da to nije
uvijek slučaj to jako često nije slučaj. Kao primjer navodim 
primjer djelovanja sile (npr. ovješen uteg) na nit. Za modeliranje
ovakvog sustava je potrebno opisati djelovanje te sile duž
cijelu nit, ali takav opis bi zahtijevao uporabu Diracove delta 
funkcije koja nikako nije neprekinuta, štoviše ona nije
ni funkcija nego distribucija.

\begin{figure}[htb]
	\centering
	\includegraphics[width=0.38\linewidth]{Figures/nit.png}
	\caption{Djelovanje točkaste sile na nit}
	\label{nit}
\end{figure}

\section{Mesh}
\label{mesh}
Mesh je particija domene na sitne dijelove s kojima ju
možemo aproksimirati. U ovoj implementaciji ti dijelovi 
će biti trokuti (jer radimo s 2D domenom), pa ću proces
stvaranja mesha često zvati i triangulacija. Ti sitni 
dijelovi koji sačinjavaju mesh se zovu elementi. 
Odavde i naziv "metoda konačnih elemenata". Točke koje
čine jedan element (npr. vrhovi trokuta koji predstavlja
jedan elemente) ću u ovom radu zvati čvorovima.

\begin{figure}[htbp]
  \centering
  \begin{subfigure}[b]{0.45\linewidth}
    \centering
    \includegraphics[width=\linewidth]{Figures/2Dmesh.png}
    \caption{Primjer 2D mesha pravokutnika}
    \label{rectMesh}
  \end{subfigure}
  \hfill
  \begin{subfigure}[b]{0.45\linewidth}
    \centering
    \includegraphics[width=\linewidth]{Figures/apple.png}
    \caption{Složeni 3D mesh jabuke}
    \label{jabuka}
  \end{subfigure}
  \caption{Primjeri mesheva}
  \label{meshevi}
\end{figure}

Bitno je napomenuti kako izbor mesha ne smije biti 
proizvoljan. Postoje kvalitetni i manje kvalitetni 
meshevi. U literaturi ne postoji nekakav jedinstveni 
kriterij koji mesh mora zadovoljavati, tako da 
je generiranje mesha u primijeni uvijek heuristički 
algoritam.
Očito je za zaključiti da bi mesh trebao pravilno popločavati
domenu tj. da ne bi smio sadržavati rupe koje u potpunosti
pripadaju domeni i da bi trebao što bolje aproksimirati domenu
kako su elementi manji odnosno kako je mesh finiji.
\bigskip
\\ 
Svakako valja napomenuti da ne želimo imati previše izdužene
elemente, stoga uvodimo sljedeći kriterij.
$$\frac{h_k}{\rho_k} \leq C$$
Gdje je $h_k$ duljina najdulje stranice trokuta, a 
$\rho$ je promjer upisane kružnice promatranog 
elementa. Razlog zašto ovakav kriterij smanjuje 
izduženost jednog elementa se lako vidi na sljedećoj
slici.
\begin{figure}[htb]
	\centering
	\includegraphics[width=0.38\linewidth]{Figures/element.png}
	\caption{najdulja stranica trokuta $h_k$ i promjer
  upisane kružnice $\rho_k$}
\end{figure}
\subsection{Bazne funkcije}

Nakon uspješno generiranog mesha je potrebno definirati
bazne funkcije. Bazne fukcije ćemo definirati tako da
na svakom elementu definiramo polinom prvog stupnja koji
iznosi $1$ iznad jednog vrha, a $0$ iznad ostalih \ref{baznaFja}
\begin{figure}[htb]
	\centering
	\includegraphics[width=0.5\linewidth]{Figures/baznaFja.png}
	\caption{bazna fukcija za neki čvor $j$}
	\label{baznaFja}
\end{figure}
Ovo ćemo napraviti za svaki čvor. 
\bigskip
\\ 
Kada budemo radili račun kao što su numerička integracija i 
slično nad elementima bit će od velike koristi transformirati
koordinate tako da element koji trenutačno računamo postane 
oblika pravokutnog trokuta s vrhovima u $(0,0)$,$(1,0)$,$(0,1)$. 
Takav trokut se u literaturi uobičajeno zove referentni element.
Neka su $(X_0, Y_0)$,$(X_1, Y_1)$,$(X_2, Y_2)$ vrhovi elementa
kojeg želimo preslikati. Supstitucija $(x, y) \rightarrow (\hat x, \hat y)$
je dana s:
\begin{align}
  x = (X_1 - X_0)\hat x + (X_2 - X_0)\hat y + X_0 \\ 
  y = (Y_1 - Y_0)\hat x +(Y_2 - Y_0) \hat y + Y_0
\end{align}
Potrebno je odrediti kako točno glase bazne funkcije nad 
referentnim elementom.
Indeksirajmo čvor $(0,0)$ s $0$, čvor $(1,0)$ s $1$a i 
čvor $(0, 1)$ s $2$.
Imamo:
\begin{align}
\varphi_0(x,y) = 1 - x - y\\
\varphi_1(x,y) = x\\
\varphi_2(x,y) = y
\end{align}
\section{Slaba formulacija problema}
\label{slabaFormulacija}
Radi daljnje analize potrebno je navesti jedan bitan 
teorem.
\\ 
\begin{teorem}[Pravilo za divergenciju produkta]
\label{tm1}
Neka je $\Omega \subset \mathbb{R}^n$ omeđen otvoreni
skup s komadno glatkom granicom $\Gamma = \partial \Omega$,
te neka su $u : \Omega \to \mathbb{R}$ i $\mathbf{V} : \Omega \to \mathbb{R}^n$ 
dovoljne regularnosti odnosno $\partial \Omega$ je
Lipschitz neprekinuta, a $u,v \in H^1(\Omega)$
. Tada vrijedi:
\[
\int_{\Gamma} u \mathbf{V} \cdot \hat{\mathbf{n}} \, d\Gamma = 
\int_{\Omega} u \, \nabla \cdot \mathbf{V} \, d\Omega + 
\int_{\Omega} \nabla u \cdot \mathbf{V} \, d\Omega.
\]
\end{teorem}
\textbf{Napomena:} $H^1(\Omega)$ označava Soboljev prostor.
\bigskip
\\ 
Vratimo se sada na \ref{PDJ}.
Cilj nam je dobiti njezinu takozvanu slabu formulaciju.
Prvo ćemo reformulirati PDJ. Primijetimo da vrijedi:
$$\nabla \cdot (A \nabla u) = 
  A \frac{\partial^2 u}{\partial x^2}
	+ B \frac{\partial^2 u}{\partial x \partial y}
	+ C \frac{\partial^2 u}{\partial y^2}
$$
Gdje:
$$M = 
\begin{pmatrix}
  C_0 & C_1 \\ 
  0 & C_2
\end{pmatrix}
 $$
 Također vrijedi:
 $$ \nabla u \cdot a =
 C_3 \frac{\partial u}{\partial x}
	+ C_4 \frac{\partial u}{\partial y} $$

Gdje:
$$a = 
\begin{pmatrix}
  C_3\\ 
  C_4
\end{pmatrix}
$$
Time dobivamo sljedeću jednadžbu:
\begin{equation}
  \label{vektorVerzija}
\nabla \cdot (M \nabla u)  + \nabla u \cdot a + C_5u = f(x,y)
\end{equation}
Do slabe formulacije dolazimo tako da obje strane \ref{vektorVerzija} 
pomnožimo
s baznom funkcijom $\varphi_i$, integriramo po $\Omega$
i onda primijenimo  \ref{tm1} 
imamo:
\begin{equation}
  \int_{\Omega}\nabla \cdot (M \nabla u) \varphi_i \, d\Omega  + 
  \int_{\Omega} \nabla u \cdot a\varphi_i \, d\Omega  + C_5\int_{\Omega}u\varphi_i\,d\Omega  =
  \int_{\Omega}f(x,y)\varphi_i \, d\Omega 
\end{equation}
Primjenom teorema \ref{tm1} 
dobije se:
\begin{equation}
  \label{kobasica}
  \int_{\partial \Omega}(M \nabla u) \varphi_i \, d\Gamma  - 
  \int_{\Omega}(M \nabla u) \cdot \nabla \varphi_i \, d\Omega  + 
  \int_{\Omega} \nabla u \cdot a\varphi_i \, d\Omega  + C_5\int_{\Omega}u\varphi_i\,d\Omega  =
  \int_{\Omega}f(x,y)\varphi_i \, d\Omega 
\end{equation}
s obzirom na to da su funkcije $\varphi_i$ po definiciji
na rubu jednake $0$, jednadžba 
\ref{kobasica} se 
svede na:
\begin{equation}
  \label{slabaFor}
  -\int_{\Omega}(M \nabla u) \cdot \nabla \varphi_i \, d\Omega  + 
  \int_{\Omega} \nabla u \cdot a\varphi_i \, d\Omega  + C_5\int_{\Omega}u\varphi_i\,d\Omega  =
  \int_{\Omega}f(x,y)\varphi_i \, d\Omega 
\end{equation}
Primijetimo kako u ovakvom obliku funkcija $f(x,y)$ 
na sebi ima puno slabije restrikcije (Sjetimo se \ref{uvjetNaf})
\bigskip
\\
Sada interpolacijom:
$$u = \sum_{j=0}^N \alpha_j \varphi_j$$
dobivamo sustav:
\begin{equation}
\label{linSustav}
A x = b
\end{equation}
Gdje:
\begin{equation}
\label{elMat}
(A)_{i,j} = \int_{\Omega} (M\nabla \varphi_j) \cdot \nabla \varphi_i \, d\Omega + 
\int_{\Omega} (\nabla \varphi_j \cdot a) \varphi_i \, d\Omega +
C_5 \int_{\Omega} \varphi_j \varphi_i \, d\Omega
\end{equation}
$$b_i = \int_{\Omega} f(x,y) \varphi_i \, d \Omega$$

Naime ovaj postupak vrijedi samo s Dirichletovim
rubnim uvjetima. Neka su čvorovi mesha na rubu indeksirani
od $N+1$ do $N_b$.
Kada bi se rješenje na rubu ponašalo kao
$R(x,y)$ gdje $R(x,y) = \sum_{j=N + 1}^{N_b} g_j \varphi_j$,
onda bismo morali napraviti sljedeću
supstituciju:
$$u = \overset{\circ}u + R$$
Sustav \ref{linSustav}
postaje:
$$Ax = b - r$$
gdje je:
$$r_i = - \int_{\Omega}(M\nabla R) \nabla \varphi_i \, d \Omega +
\int_{\Omega}(\nabla R \cdot a) \varphi_i \, d \Omega +
C_5 \int_{\Omega} R \varphi_i\, d\Omega
$$
Time smo problem rješavanja PDJ sveli na problem
rješavanja linearnog sustava.
%napiši opet isto, ali za rubni uvjet
%eventualno možeš pričati o proširenjima za vrijeme
\section{Postojanje rješenja}
\label{postojanje}
%Lax Milgram
Iako se možda na prvi pogled čini da smo gotovi, potrebno 
je vidjeti da rješenje uopće postoji, stoga uvodim 
Lax-Milgramov teorem.
\begin{teorem}
  \label{laxMil}
Neka je $ V $ Hilbertov prostor i neka je $ a: V \times V \rightarrow \mathbb{R} $ bilinearna forma koja zadovoljava:
\begin{itemize}
    \item \textbf{Kontinuitet:} postoji konstanta $ C > 0 $ takva da
    $$
    |a(u, v)| \leq C \|u\|_V \|v\|_V \quad \text{za sve } u, v \in V.
    $$
    
    \item \textbf{Koercitivnost (eliptičnost):} postoji konstanta $ \alpha > 0 $ takva da
    $$
    a(v, v) \geq \alpha \|v\|_V^2 \quad \text{za sve } v \in V.
    $$
\end{itemize}

Tada za svaki linearni funkcional $ f \in V' $ postoji jedinstveno $ u \in V $ takvo da:
$$
a(u, v) = f(v) \quad \text{za sve } v \in V.
$$
\end{teorem}
Lako se vidi da se naš sustav \ref{linSustav} 
može svesti na problem s linearnim funkcionalima iz
\ref{laxMil}. Iz toga se lako zaključuje da
postojanje rješenja možemo garantirati samo za ispravan mesh
(to garantira implementacija) i za eliptičke PDJ što se lako provjeri programski.

\section{Implementacijski detalji}
\label{implementacijaOpis}
Sada kada je jasna matematička pozadina 
Galjorkinove metode konačnih elemenata potrebno je 
obratiti pažnju na implementaciju i implementacijske detalje.
s obzirom na to da je implementacija izvedena u C++-u postoji
dosta različitih optimizacija koje se mogu implementirati,
stoga ih navodim u ovom poglavlju.

\subsection{Generiranje mesha}

S obzirom na to da
je jednostavnost korištenja ključno svojstvo ove implementacije 
odlučio sam se za sljedeći način unosa mesha u računalo.
\bigskip
\\ 
\textbf{Korisnik definira funkciju koja prima koordinate, a
vraća vrijednost "true" ako je točka unutar domene ili
"false" ako je točka izvan domene}
\bigskip
\\ 
Time smo implicitno definirali domenu PDJ-a.
Ovakav pristup omogućava da se domena definira na sličan 
način kako bi to radili na papiru iz čega proizlazi jednostavnost.
Druge C++ biblioteke za unos mesha uglavnom koriste .msh datoteke.
Ovakav pristup je dobar za komplicirane domene koje se ne mogu
lako definirati s implicitnim funkcijama, ali je presložen za neke
jednostavnije primjene. Valja napomenuti da neke biblioteke 
daju mogućnost da se domena definira preko ključnih riječi kao 
što su "circle" ili "rectangle". U usporedbi s tim rješenjima 
moj pristup daje veću fleksibilnost.
\bigskip
\\ 
Jednostavan način korištenja povlači i težinu implementacije.
Naime, kvalitetan mesh može značajno smanjiti numeričku grešku
rješenja. Naravno greška se uvijek može smanjiti tako da 
učinimo mesh finijim, ali to ne možemo raditi u nedogled.
Ova implementacija mesh generira na poprilično jednostavan način.
Za ulaz ona traži pravokutnik koji omeđuje domenu nad kojom
rješavamo PDJ. Taj pravokutnik se potom triangulira i za mesh
se uzimaju samo elementi čija su sva tri vrha unutar domene.
Ovakav pristup daje nekvalitetan mesh na rubu. Istraživajući
bolje pristupe generiranju mesha otkrio sam razne knjige koje
se bave ovom problematikom. Došao sam do zaključka da 
je ovaj problem previše složen za ovaj rad i stoga smatram da
je za sada bitno napraviti implementaciju koja se može naknadno lako 
nadograditi.
\\
Iako jednostavan pristup ovom problemu funkcionira
mislim da je bitno naglasiti gdje ova implementacija ima
mjesta za napredak.

\subsection{CSR reprezentacija matrice}
Matrica $A$ u \ref{linSustav} 
često sadrži puno nul elemenata. To se vrlo lako može 
vidjeti iz načina na koji se ona računa. Naime, $(A)_{i,j}$ 
neće biti jednak $0$ samo kada su odgovarajući čvorovi 
mesha ($i$ i $j$) susjedni. S obzirom na to da je 
zbog veće preciznosti poželjno da
matrice sustava budu što veće uvodi se CSR (eng. compact sparse row)
reprezentacija matrice. Ovaj pojam se može lako objasniti na
sljedećem primjeru.
$$M = \begin{pmatrix}
  a & 0 & f & 0 & g\\ 
  0 & b & k & m & 0\\ 
  h & l & c & 0 & r\\ 
  0 & n & 0 & d & p\\ 
  i & 0 & s & q & e
\end{pmatrix}$$
Na slici \ref{CSR} 
se vidi matrica $M$ zapisana preko CSR. Vidimo da se
ova struktura podataka sastoji od 3 vektora. 
Vektori $A$ i $C$ su jednake veličine kao broj ne nul 
elemenata, a svaki element $R$ odgovara jednom retku.
Matrica $A$ se stvara tako da čitajući redak po redak
s lijeva na desno umećemo elemente matrice $M$ koji 
nisu jednaki $0$. Vrijednost $C_i$ odgovara brojčanoj
vrijednosti stupca u kojem se element $A_i$ nalazi 
a element $R_i$ govori od kojeg indeksa kreće $i-ti$
redak matrice $M$ u vektoru $C$ odnosno $A$. 
\\ 
\bigskip
Dohvat elementa $i,j$ ovakve matrice se vrlo lako implementira.
\begin{verbatim}
inline ld get(ull i, ull j) {
  ull it = R[i];
  while (it < R[i + 1] && C[it] < j) it++;
  if (C[it] == j) {
    return A[it];
  }
  return 0;
}
\end{verbatim}
Pažljivi čitatelj će primijeti da se pri traženju odgovarajućeg
stupca (While petlja) radi linearno pretraživanje. Ovaj dio 
se može ubrzati binarnim pretraživanjem, ali s obzirom na 
izuzetno malen broj ne nul elementa u jednom stupcu (maksimalno $6$)
u trenutačnoj implementaciji mesha, upitno je koliko veliko ubrzanje 
bi se ostvarilo takvom implementacijom.
\\ 
\bigskip
Slično se implementira i mijenjanje nekog ne nul elementa.
\begin{verbatim}
inline void inc(ull i, ull j, ld val) {
  ull it = R[i];
  while (it < R[i + 1] && C[it] < j) it++;
  if (C[it] == j) {
    A[it] += val;
  }
}
\end{verbatim}
\begin{figure}[htb]
	\centering
	\includegraphics[width=0.7\linewidth]{Figures/CSR.png}
	\caption{grafički prikaz CSR reprezentacije matrice $A$}
	\label{CSR}
\end{figure}

\subsection{Numerička integracija}
Za numeričku integraciju je korištena metoda Gaussovih 
čvorova odnosno za referentni element $\hat{K}$ integral
po njemu se može zapisati kao.
$$\int_{\hat{K}} f(x,y) \, d\hat{K} = \sum_{i = 0}^n w_i f(x_i,y_i)$$
Gdje su trojke $(x_i, y_i, w_i)$ izračunate unaprijed.
Moja implementacija koristi sljedeće Gaussove čvorove:
$(0,0,1/3), (1,0, 1/3),(0,1,1 / 3)$


\subsection{Rješavanje sustava}

Računski najsloženiji dio je svakako rješavanje linearnog 
sustava. Na raspolaganju je puno različitih algoritama različitih 
složenosti, pa tako su neki prilagođeniji nekoj specifičnoj PDJ.
Naprimjer Choleskyjeva faktorizacija bi mogla biti korištena kada
bih mogao garantirati da će matrica $A$ biti simetrična pozitivno 
definitna, ali to se iz \ref{elMat} 
lako vidi da nije slučaj.
\bigskip
\\ 
Daljnjom analizom se može zaključiti da direktne metode nisu dobar
izbor jer uz vremenski skupo plaćanje faktorizacija matrice (QR, LU \dots),
takve matrice mogu dovesti do pojavljivanja dodatnih ne nul elemenata 
koji bi povećali zahtjeve za memorijom računala.
\bigskip
\\ 
Preostaju nam iterativne metode i one se uobičajeno i koriste kod 
rada s rijetko popunjenim matricama. Neke od mogućih metoda su 
Jacobijeva, Gauss-Seidel, SOR, konjugirani gradijenti, GMRES.
\bigskip
\\
Radi jednostavnosti ova implementacija koristi Gauss-Seidel-ovu metodu 
koja neće nužno konvergirati za svaku moguću PDJ koju će korisnik unijeti,
ali će biti dovoljno dobra za standardne probleme nad 2D meshevima.
U nastavku dajem pseudokod Gauss-Seidela:
\begin{verbatim}
##rješavanje sustava Ax = b Gauss-Seidelom
Uzimamo neki vektor x duljine n
za svaku iteraciju k od brojIteracija
  za svaki redak i matrice A
    ##računamo novi xi
    sumaManjih = 0
    za j od 0 do i 
      sumaManjih += Aij* xj
    sumaVecih = 0
    za j od i + 1 do n
      sumaVecih += Aij * xj
    xi = (bi - sumaManjih - sumaVecih) / Aii 
\end{verbatim}
Dodatno Gauss-Seidel se može ubrzati tako da iskoristimo rijetku popunjenost
matrice. Naime najdublje ugniježdene petlje se ne moraju iterirati po svim elementima
nego samo po onima koji nisu jednaki $0$. Pa tako možemo iskoristiti CSR kako
bi dobili i vremensku efikasnost. Petlja po iteracijama Gauss-Seidela u C++ glasi:
\begin{verbatim}
for (u k = 1; k <= numOfIterations; k++) {
  for (u i = 0; i < M.numOfRows; i++) {
    ld sumOfLessThan_i = 0;
    u j;
    for (j = M.R[i];
        j < M.R[i + 1] && M.C[j] < i;
        j++) {
      sumOfLessThan_i += M.A[j] * x[M.C[j]];
    }
    ld sumOfGreaterThan_i = 0;
    ld Mii = 0;
    if (M.C[j] == i) {
      Mii = M.A[j];
      j++;
    } 
    for (; j < M.R[i + 1]; j++) {
      sumOfGreaterThan_i += M.A[j] * x[M.C[j]];
    }
    x[i] = 
      (b[i] - sumOfLessThan_i - sumOfGreaterThan_i) / Mii; 
  }
\end{verbatim}
Ovom optimizacijom smo sveli vremensku složenost s 
$\mathcal{O}(k n^2)$ na $\mathcal{O}(k \cdot 6n)$ odnosno na $\mathcal{O}(k n)$.
\bigskip
\\
Kao i prije navodim gdje se implementacija može poboljšati. Istraživanjem 
sam dobio dojam da je GMRES metoda bolji izbor. Također bi bila
potencijalno dobra 
ideja da se provjerava simetričnost i pozitivna definitnost matrice 
kako bi se primijenila brža metoda koja ovisi o tim svojstvima 
matrice.

\subsection{Prevođenje koda}
\textit{Ovdje ću malo prokomentirati koje sam točno zastavice
i opcije koristio pri prevođenju moje implementacije.}


%-------------------------------------------------------------------------------
\chapter{Rezultati i rasprava}
\label{pog:rezultati_i_rasprava}



%--- ZAKLJUČAK / CONCLUSION ----------------------------------------------------
\chapter{Zaključak}
\label{pog:zakljucak}



%--- LITERATURA / REFERENCES ---------------------------------------------------

% Literatura se automatski generira iz zadane .bib datoteke / References are automatically generated from the supplied .bib file
% Upiši ime BibTeX datoteke bez .bib nastavka / Enter the name of the BibTeX file without .bib extension
\bibliography{literatura}



%--- SAŽETAK / ABSTRACT --------------------------------------------------------

% Sažetak na hrvatskom
\begin{sazetak}
	Unesite sažetak na hrvatskom.

\end{sazetak}

\begin{kljucnerijeci}
	prva ključna riječ; druga ključna riječ; treća ključna riječ
\end{kljucnerijeci}


% Abstract in English
\begin{abstract}
	Enter the abstract in English.

\end{abstract}

\begin{keywords}
	the first keyword; the second keyword; the third keyword
\end{keywords}


%--- PRIVITCI / APPENDIX -------------------------------------------------------

% Sva poglavlje koja slijede će biti označena slovom i riječi privitak / All following chapters will be denoted with an appendix and a letter
\backmatter

\chapter{The Code}



\end{document}
